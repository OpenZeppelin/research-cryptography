\documentclass[dvipsnames]{beamer}
\usetheme{Copenhagen}

\usepackage{epigraph}
\usepackage{xcolor}

\setlength\epigraphwidth{.8\textwidth}
\setlength\epigraphrule{0pt}

\newcommand{\pub}[1]{\textcolor{blue}{\texttt{#1}}}
\newcommand{\priv}[1]{\textcolor{brown}{\texttt{#1}}}
\newcommand{\enc}[1]{\colorbox{RubineRed}{\textcolor{white}{\texttt{#1}}}}

% Add a title page for each section and subsection
\AtBeginSection[]{
    \begin{frame}
        \frametitle{Table of Contents}
        \tableofcontents[currentsection]
    \end{frame}
}


\begin{document}
    \title{SNARKS}
    \date{}

    \begin{frame}
        \maketitle

        \epigraph{I ask the fundamental question of rationality: Why do you believe what you believe? What do you think you know and how do you think you know it?}{\textit{Harry Potter\\Harry Potter and the methods of rationality}}
    \end{frame}

    \section{A simple zkARK}

    \begin{frame}{Setup}
        \begin{itemize}
            \item Choose a generator \pub{g} and prime \pub{p}
            \item Choose a private key \priv{x}
            \item Calculate the public key \pub{g\priv{$^x$} mod p}
            \item From now on the notation \enc{x} denotes this ``encryption'' operation

        \end{itemize}

        \begin{block}{}
            The goal is to \textbf{demonstrate knowledge} of \priv{x} without revealing it.
        \end{block}
    \end{frame}

    \begin{frame}{Mathematical tools}
        \begin{block}{Perfect hiding}
        If \priv{r} is a secret random value, \pub{\priv{x} + \priv{r} mod n} is perfectly hidden
        \end{block}

        \begin{block}{Partially homomorphic encryption}
            \begin{itemize}
                \item   \enc{a}$^c$ = \pub{(g\priv{$^a$})$^c$ mod p}
                        \\\hspace{0.8cm}= \pub{g\priv{$^{c\cdot a}$} mod p}
                        \\\hspace{0.8cm}= \enc{c$\cdot$a}
                \item   \enc{a} * \enc{b} = \pub{g\priv{$^a$} mod p} * \pub{g\priv{$^b$} mod p}
                        \\\hspace{1.3cm}= \pub{g\priv{$^{a+b}$} mod p}
                        \\\hspace{1.3cm}= \enc{a+b}
            \end{itemize}
        \end{block}
    \end{frame}

    \begin{frame}{The Protocol}
        \begin{itemize}
            \item \enc{x} is common knowledge.
            \item Alice claims to know \priv{x}
        \end{itemize}
        \vspace{1cm}

        \begin{columns}
            \begin{column}{0.5\textwidth}
                \textbf{Alice}
                \begin{itemize}
                    \item Generates \priv{r}. Sends \enc{r}
                    \item[]
                    \item Sends \pub{\priv{x} + c$\cdot$r}
                    \item[]
                \end{itemize}
            \end{column}
            \begin{column}{0.5\textwidth}  %%<--- here
                \textbf{Bob}
                \begin{itemize}
                    \item[]
                    \item Generates and sends \pub{c}
                    \item[]
                    \item Calculates \enc{x + c$\cdot$r} ($\times$2)
                \end{itemize}
            \end{column}
        \end{columns}
    \end{frame}


\end{document}

